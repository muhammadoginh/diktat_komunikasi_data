\chapter{Pengantar komunikasi Data}
Data communications and networking are changing the way we do business and the way we live. Business decisions have to be made ever more quickly, and the decision makers require immediate access to accurate information. Why wait a week for that report from Germany to arrive by mail when it could appear almost instantaneously through computer networks? Businesses today rely on computer networks and internetworks. But before we ask how quickly we can get hooked up, we need to know how networks operate, what types of technologies are available, and which design best fills which set of needs. The development of the personal computer brought about tremendous changes for business, industry, science, and education. A similar revolution is occurring in data communications and networking. Technological advances are making it possible for communications links to carry more and faster signals. As a result, services are evolving to allow use of this expanded capacity. For example, established telephone services such as conference calling, call waiting, voice mail, and caller ID have been extended. Research in data communications and networking has resulted in new technologies. One goal is to be able to exchange data such as text, audio, and video from all points in the world. We want to access the Internet to download and upload information quickly and accurately and at any time. This chapter addresses four issues: data communications, networks, the Internet, and protocols and standards. First we give a broad definition of data communications. Then we define networks as a highway on which data can travel. The Internet is discussed as a good example of an internetwork (i.e., a network of networks). Finally, we discuss different types of protocols, the difference between protocols and standards, and the organizations that set those standards.

Setelah membaca buku ini, pembaca diharapkan dapat:
\begin{itemize}
  \item[$\star$] Menjelaskan komponen komunikasi data 
  \item[$\star$] Menjelaskan model jaringan
  \item[$\star$] Menjelaskan Perbedaan Data dan sinyal
  \item[$\star$] Menjelaskan transmisi digital dan analog
\end{itemize}

\section{Komunikasi Data}
When we communicate, we are sharing information. This sharing can be local or remote. Between individuals, local communication usually occurs face to face, while remote communication takes place over distance. The term telecommunication, which includes telephony, telegraphy, and television, means communication at a distance (tele is Greek for "far"). The word data refers to information presented in whatever form is agreed upon by the parties creating and using the data. Data communications are the exchange of data between two devices via some form of transmission medium such as a wire cable. For data communications to occur, the communicating devices must be part of a communication system made up of a combination of hardware (physical equipment) and software (programs). The effectiveness of a data communications system depends on four fundamental characteristics: delivery, accuracy, timeliness, and jitter.
\begin{enumerate}
  \item Delivery. The system must deliver data to the correct destination. Data must be received by the intended device or user and only by that device or user.
  \item Accuracy. The system must deliver the data accurately. Data that have been altered in transmission and left uncorrected are unusable
  \item Timeliness. The system must deliver data in a timely manner. Data delivered late are useless. In the case of video and audio, timely delivery means delivering data as they are produced, in the same order that they are produced, and without significant delay. This kind of delivery is called real-time transmission
  \item Jitter. Jitter refers to the variation in the packet arrival time. It is the uneven delay in the delivery of audio or video packets. For example, let us assume that video packets are sent every 3D ms. Ifsome of the packets arrive with 3D-ms delay and others with 4D-ms delay, an uneven quality in the video is the result.
\end{enumerate}

\subsection{Components}
A data communications system has five components (see Figure 1.1).
\begin{enumerate}
  \item Message. The message is the information (data) to be communicated. Popular forms of information include text, numbers, pictures, audio, and video.
  \item Sender. The sender is the device that sends the data message. It can be a computer, workstation, telephone handset, video camera, and so on.
  \item Receiver. The receiver is the device that receives the message. It can be a computer, workstation, telephone handset, television, and so on.
  \item Transmission medium. The transmission medium is the physical path by which a message travels from sender to receiver. Some examples of transmission media include twisted-pair wire, coaxial cable, fiber-optic cable, and radio waves.
  \item Protocol. A protocol is a set of rules that govern data communications. It represents an agreement between the communicating devices. Without a protocol, two devices may be connected but not communicating, just as a person speaking French cannot be understood by a person who speaks only Japanese.
\end{enumerate}

\subsection{Data Representation}
Information today comes in different forms such as text, numbers, images, audio, and video.

\subsection*{Text}
In data communications, text is represented as a bit pattern, a sequence of bits (Os or Is). Different sets of bit patterns have been designed to represent text symbols. Each set is called a code, and the process of representing symbols is called coding. Today, the prevalent coding system is called Unicode, which uses 32 bits to represent a symbol or character used in any language in the world. The American Standard Code for Information Interchange (ASCII), developed some decades ago in the United States, now constitutes the first 127 characters in Unicode and is also referred to as Basic Latin. Appendix A includes part of the Unicode

\subsection*{Numbers}
Numbers are also represented by bit patterns. However, a code such as ASCII is not used to represent numbers; the number is directly converted to a binary number to simplify mathematical operations. Appendix B discusses several different numbering systems.

\subsection*{Images}
Images are also represented by bit patterns. In its simplest form, an image is composed of a matrix of pixels (picture elements), where each pixel is a small dot. The size of the pixel depends on the resolution. For example, an image can be divided into 1000 pixels or 10,000 pixels. In the second case, there is a better representation of the image (better resolution), but more memory is needed to store the image. After an image is divided into pixels, each pixel is assigned a bit pattern. The size and the value of the pattern depend on the image. For an image made of only black-and-white dots (e.g., a chessboard), a I-bit pattern is enough to represent a pixel. If an image is not made of pure white and pure black pixels, you can increase the size of the bit pattern to include gray scale. For example, to show four levels of gray scale, you can use 2-bit patterns. A black pixel can be represented by 00, a dark gray pixel by 01, a light gray pixel by 10, and a white pixel by 11. There are several methods to represent color images. One method is called RGB, so called because each color is made of a combination of three primary colors: red, green, and blue. The intensity of each color is measured, and a bit pattern is assigned to it. Another method is called YCM, in which a color is made of a combination of three other primary colors: yellow, cyan, and magenta.

\subsection*{Audio}
Audio refers to the recording or broadcasting of sound or music. Audio is by nature different from text, numbers, or images. It is continuous, not discrete. Even when we use a microphone to change voice or music to an electric signal, we create a continuous signal. In Chapters 4 and 5, we learn how to change sound or music to a digital or an analog signal.

\subsection*{Video}
Video refers to the recording or broadcasting of a picture or movie. Video can either be produced as a continuous entity (e.g., by a TV camera), or it can be a combination of images, each a discrete entity, arranged to convey the idea of motion. Again we can change video to a digital or an analog signal, as we will see in Chapters 4 and 5.

\subsection{Data Flow}
Communication between two devices can be simplex, half-duplex, or full-duplex as shown in Figure 1.2.

\subsection*{Simplex}
In simplex mode, the communication is unidirectional, as on a one-way street. Only one of the two devices on a link can transmit; the other can only receive (see Figure 1.2a). Keyboards and traditional monitors are examples of simplex devices. The keyboard can only introduce input; the monitor can only accept output. The simplex mode can use the entire capacity of the channel to send data in one direction

\subsection*{Half-Duplex}
In half-duplex mode, each station can both transmit and receive, but not at the same time. : When one device is sending, the other can only receive, and vice versa (see Figure 1.2b).

The half-duplex mode is like a one-lane road with traffic allowed in both directions. When cars are traveling in one direction, cars going the other way must wait. In a half-duplex transmission, the entire capacity of a channel is taken over by whichever of the two devices is transmitting at the time. Walkie-talkies and CB (citizens band) radios are both half-duplex systems. The half-duplex mode is used in cases where there is no need for communication in both directions at the same time; the entire capacity of the channel can be utilized for each direction.

\subsection*{Full-Duplex}
In full-duplex m.,lle (als@ called duplex), both stations can transmit and receive simultaneously (see Figure 1.2c). The full-duplex mode is like a tW<D-way street with traffic flowing in both directions at the same time. In full-duplex mode, si~nals going in one direction share the capacity of the link: with signals going in the other din~c~on. This sharing can occur in two ways: Either the link must contain two physically separate t:nmsmissiIDn paths, one for sending and the other for receiving; or the capacity of the ch:arillilel is divided between signals traveling in both directions. One common example of full-duplex communication is the telephone network. When two people are communicating by a telephone line, both can talk and listen at the same time. The full-duplex mode is used when communication in both directions is required all the time. The capacity of the channel, however, must be divided between the two directions.

\section{Jaringan}
A network is a set of devices (often referred to as nodes) connected by communication links. A node can be a computer, printer, or any other device capable of sending and/or receiving data generated by other nodes on the network.

\subsection{Distributed Processing}
Most networks use distributed processing, in which a task is divided among multiple computers. Instead of one single large machine being responsible for all aspects of a process, separate computers (usually a personal computer or workstation) handle a subset.

\subsection{Network Criteria}
A network must be able to meet a certain number of criteria. The most important of these are performance, reliability, and security.

\subsection*{Performance}
Performance can be measured in many ways, including transit time and response time. Transit time is the amount of time required for a message to travel from one device to another. Response time is the elapsed time between an inquiry and a response. The performance of a network depends on a number of factors, including the number of users, the type of transmission medium, the capabilities of the connected hardware, and the efficiency of the software.

Performance is often evaluated by two networking metrics: throughput and delay. We often need more throughput and less delay. However, these two criteria are often contradictory. If we try to send more data to the network, we may increase throughput but we increase the delay because of traffic congestion in the network.

\subsection*{Reliability}
In addition to accuracy of delivery, network reliability is measured by the frequency of failure, the time it takes a link to recover from a failure, and the network's robustness in a catastrophe.

\subsection*{Security}
Network security issues include protecting data from unauthorized access, protecting data from damage and development, and implementing policies and procedures for recovery from breaches and data losses.

\subsection{Physical Structures}
Before discussing networks, we need to define some network attributes.

\subsection*{Type of Connection}
A network is two or more devices connected through links. A link is a communications pathway that transfers data from one device to another. For visualization purposes, it is simplest to imagine any link as a line drawn between two points. For communication to occur, two devices must be connected in some way to the same link at the same time. There are two possible types of connections: point-to-point and multipoint.

\textbf{Point-to-Point} A point-to-point connection provides a dedicated link between two devices. The entire capacity of the link is reserved for transmission between those two devices. Most point-to-point connections use an actual length of wire or cable to connect the two ends, but other options, such as microwave or satellite links, are also possible (see Figure 1.3a). When you change television channels by infrared remote control, you are establishing a point-to-point connection between the remote control and the television's control system.

\textbf{Multipoint} A multipoint (also called multidrop) connection is one in which more than two specific devices share a single link (see Figure 1.3b). In a multipoint environment, the capacity of the channel is shared, either spatially or temporally. Ifseveral devices can use the link simultaneously, it is a spatially shared connection. If users must take turns, it is a timeshared connection.

\subsection*{Physical Topology}
The term physical topology refers to the way in which a network is laid out physically.: Two or more devices connect to a link; two or more links form a topology. The topology of a network is the geometric representation of the relationship of all the links and linking devices (usually called nodes) to one another. There are four basic topologies possible: mesh, star, bus, and ring (see Figure 1.4). 

\textbf{Mesh} In a mesh topology, every device has a dedicated point-to-point link to every other device. The term dedicated means that the link carries traffic only between the two devices it connects. To find the number of physical links in a fully connected mesh network with n nodes, we first consider that each node must be connected to every other node. Node 1 must be connected to n - I nodes, node 2 must be connected to n - 1 nodes, and finally node n must be connected to n - 1 nodes. We need n(n - 1) physical links. However, if each physical link allows communication in both directions (duplex mode), we can divide the number of links by 2. In other words, we can say that in a mesh topology, we need

\begin{equation*}
  n(n - 1)/2
\end{equation*}

\noindent dupleks-mode links.

To accommodate that many links, every device on the network must have n - 1 input/output (VO) ports (see Figure 1.5) to be connected to the other n - 1 stations.

A mesh offers several advantages over other network topologies. First, the use of dedicated links guarantees that each connection can carry its own data load, thus eliminating the traffic problems that can occur when links must be shared by multiple devices. Second, a mesh topology is robust. If one link becomes unusable, it does not incapacitate the entire system. Third, there is the advantage of privacy or security. When every message travels along a dedicated line, only the intended recipient sees it. Physical boundaries prevent other users from gaining access to messages. Finally, point-to-point links make fault identification and fault isolation easy. Traffic can be routed to avoid links with suspected problems. This facility enables the network manager to discover the precise location of the fault and aids in finding its cause and solution.

The main disadvantages of a mesh are related to the amount of cabling and the number of I/O ports required. First, because every device must be connected to every other device, installation and reconnection are difficult. Second, the sheer bulk of the wiring can be greater than the available space (in walls, ceilings, or floors) can accommodate. Finally, the hardware required to connect each link (I/O ports and cable) can be prohibitively expensive. For these reasons a mesh topology is usually implemented in a limited fashion, for example, as a backbone connecting the main computers of a hybrid network that can include several other topologies.

One practical example of a mesh topology is the connection of telephone regional offices in which each regional office needs to be connected to every other regional office.

\textbf{Star Topology} In a star topology, each device has a dedicated point-to-point link only to a central controller, usually called a hub. The devices are not directly linked to one another. Unlike a mesh topology, a star topology does not allow direct traffic between devices. The controller acts as an exchange: If one device wants to send data to another, it sends the data to the controller, which then relays the data to the other connected device (see Figure 1.6) . A star topology is less expensive than a mesh topology. In a star, each device needs only one link and one I/O port to connect it to any number of others. This factor also makes it easy to install and reconfigure. Far less cabling needs to be housed, and additions, moves, and deletions involve only one connection: between that device and the hub.

Other advantages include robustness. If one link fails, only that link is affected. All other links remain active. This factor also lends itself to easy fault identification and fault isolation. As long as the hub is working, it can be used to monitor link problems and bypass defective links.

One big disadvantage of a star topology is the dependency of the whole topology on one single point, the hub. If the hub goes down, the whole system is dead. Although a star requires far less cable than a mesh, each node must be linked to a central hub. For this reason, often more cabling is required in a star than in some other topologies (such as ring or bus). The star topology is used in local-area networks (LANs). High-speed LANs often use a star topology with a central hub.

\textbf{Bus Topology} The preceding examples all describe point-to-point connections. A bus topology, on the other hand, is multipoint. One long cable acts as a backbone to link all the devices in a network (see Figure 1.7)

Nodes are connected to the bus cable by drop lines and taps. A drop line is a connection running between the device and the main cable. A tap is a connector that either splices into the main cable or punctures the sheathing of a cable to create a contact with the metallic core. As a signal travels along the backbone, some ofits energy is transformed into heat. Therefore, it becomes weaker and weaker as it travels farther and farther. For this reason there is a limit on the number of taps a bus can support and on the distance between those taps.

Advantages of a bus topology include ease of installation. Backbone cable can be laid along the most efficient path, then connected to the nodes by drop lines of various lengths. In this way, a bus uses less cabling than mesh or star topologies. In a star, for example, four network devices in the same room require four lengths of cable reaching all the way to the hub. In a bus, this redundancy is eliminated. Only the backbone cable stretches through the entire facility. Each drop line has to reach only as far as the nearest point on the backbone

Disadvantages include difficult reconnection and fault isolation. A bus is usually designed to be optimally efficient at installation. It can therefore be difficult to add new devices. Signal reflection at the taps can cause degradation in quality. This degradation can be controlled by limiting the number and spacing of devices connected to a given length of cable. Adding new devices may therefore require modification or replacement of the backbone.

In addition, a fault or break in the bus cable stops all transmission, even between devices on the same side of the problem. The damaged area reflects signals back in the direction of origin, creating noise in both directions. Bus topology was the one of the first topologies used in the design of early localarea networks. Ethernet LANs can use a bus topology, but they are less popular now.

\textbf{Ring Topology} In a ring topology, each device has a dedicated point-to-point connection with only the two devices on either side of it. A signal is passed along the ring in one direction, from device to device, until it reaches its destination. Each device in the ring incorporates a repeater. When a device receives a signal intended for another device, its repeater regenerates the bits and passes them along (see Figure 1.8).

A ring is relatively easy to install and reconfigure. Each device is linked to only its immediate neighbors (either physically or logically). To add or delete a device requires changing only two connections. The only constraints are media and traffic considerations (maximum ring length and number of devices). In addition, fault isolation is simplified. Generally in a ring, a signal is circulating at all times. If one device does not receive a signal within a specified period, it can issue an alarm. The alarm alerts the network operator to the problem and its location. However, unidirectional traffic can be a disadvantage. In a simple ring, a break in the ring (such as a disabled station) can disable the entire network. This weakness can be solved by using a dual ring or a switch capable of closing off the break.

Ring topology was prevalent when IBM introduced its local-area network Token Ring. Today, the need for higher-speed LANs has made this topology less popular.

\textbf{Hybrid Topology} A network can be hybrid. For example, we can have a main star topology with each branch connecting several stations in a bus topology as shown in Figure 1.9

\subsection{Network Models}
Computer networks are created by different entities. Standards are needed so that these heterogeneous networks can communicate with one another. The two best-known standards are the OSI model and the Internet model. In Chapter 2 we discuss these two models. The OSI (Open Systems Interconnection) model defines a seven-layer network; the Internet model defines a five-layer network. This book is based on the Internet model with occasional references to the OSI model.

\subsection{Categories of Networks}
Today when we speak of networks, we are generally referring to two primary categories: local-area networks and wide-area networks. The category into which a network falls is determined by its size. A LAN normally covers an area less than 2 mi; a WAN can be worldwide. Networks of a size in between are normally referred to as metropolitanarea networks and span tens of miles.

\subsection*{Local Area Network}
A local area network (LAN) is usually privately owned and links the devices in a single office, building, or campus (see Figure 1.10). Depending on the needs of an organization and the type of technology used, a LAN can be as simple as two PCs and a printer in someone's home office; or it can extend throughout a company and include audio and video peripherals. Currently, LAN size is limited to a few kilometers.

LANs are designed to allow resources to be shared between personal computers or workstations. The resources to be shared can include hardware (e.g., a printer), software (e.g., an application program), or data. A common example of a LAN, found in many business environments, links a workgroup of task-related computers, for example, engineering workstations or accounting PCs. One of the computers may be given a large capacity disk drive and may become a server to clients. Software can be stored on this central server and used as needed by the whole group. In this example, the size of the LAN may be determined by licensing restrictions on the number of users per copy ofsoftware, or by restrictions on the number of users licensed to access the operating system. In addition to size, LANs are distinguished from other types of networks by their transmission media and topology. In general, a given LAN will use only one type of transmission medium. The most common LAN topologies are bus, ring, and star. Early LANs had data rates in the 4 to 16 megabits per second (Mbps) range. Today, however, speeds are normally 100 or 1000 Mbps. LANs are discussed at length in Chapters 13, 14, and 15. Wireless LANs are the newest evolution in LAN technology. We discuss wireless LANs in detail in Chapter 14.

\subsection*{Wide Area Network}
A wide area network (WAN) provides long-distance transmission of data, image, audio, and video information over large geographic areas that may comprise a country, a continent, or even the whole world. In Chapters 17 and 18 we discuss wide-area networks in greater detail. A WAN can be as complex as the backbones that connect the Internet or as simple as a dial-up line that connects a home computer to the Internet. We normally refer to the first as a switched WAN and to the second as a point-to-point WAN (Figure 1.11). The switched WAN connects the end systems, which usually comprise a router (internetworking connecting device) that connects to another LAN or WAN. The point-to-point WAN is normally a line leased from a telephone or cable TV provider that connects a home computer or a small LAN to an Internet service provider (lSP). This type of WAN is often used to provide Internet access.

An early example of a switched WAN is X.25, a network designed to provide connectivity between end users. As we will see in Chapter 18, X.25 is being gradually replaced by a high-speed, more efficient network called Frame Relay. A good example of a switched WAN is the asynchronous transfer mode (ATM) network, which is a network with fixed-size data unit packets called cells. We will discuss ATM in Chapter 18. Another example ofWANs is the wireless WAN that is becoming more and more popular. We discuss wireless WANs and their evolution in Chapter 16.

\subsection*{Metropolitan Area Networks}
A metropolitan area network (MAN) is a network with a size between a LAN and a WAN. It normally covers the area inside a town or a city. It is designed for customers who need a high-speed connectivity, normally to the Internet, and have endpoints spread over a city or part of city. A good example of a MAN is the part of the telephone company network that can provide a high-speed DSL line to the customer. Another example is the cable TV network that originally was designed for cable TV, but today can also be used for high-speed data connection to the Internet. We discuss DSL lines and cable TV networks in Chapter 9.

\subsection{Interconnection of Networks: Internetwork}
Today, it is very rare to see a LAN, a MAN, or a LAN in isolation; they are connected to one another. When two or more networks are connected, they become an internetwork, or internet

As an example, assume that an organization has two offices, one on the east coast and the other on the west coast. The established office on the west coast has a bus topology LAN; the newly opened office on the east coast has a star topology LAN. The president of the company lives somewhere in the middle and needs to have control over the company from her horne. To create a backbone WAN for connecting these three entities (two LANs and the president's computer), a switched WAN (operated by a service provider such as a telecom company) has been leased. To connect the LANs to this switched WAN, however, three point-to-point WANs are required. These point-to-point WANs can be a high-speed DSL line offered by a telephone company or a cable modern line offered by a cable TV provider as shown in Figure 1.12.

\section{Internet}
The Internet has revolutionized many aspects of our daily lives. It has affected the way we do business as well as the way we spend our leisure time. Count the ways you've used the Internet recently. Perhaps you've sent electronic mail (e-mail) to a business associate, paid a utility bill, read a newspaper from a distant city, or looked up a local movie schedule-all by using the Internet. Or maybe you researched a medical topic, booked a hotel reservation, chatted with a fellow Trekkie, or comparison-shopped for a car. The Internet is a communication system that has brought a wealth of information to our fingertips and organized it for our use. The Internet is a structured, organized system. We begin with a brief history of the Internet. We follow with a description of the Internet today.

\subsection{Sejarah Singkat}
A network is a group of connected communicating devices such as computers and printers. An internet (note the lowercase letter i) is two or more networks that can communicate with each other. The most notable internet is called the Internet (uppercase letter I), a collaboration of more than hundreds of thousands of interconnected networks. Private individuals as well as various organizations such as government agencies, schools, research facilities, corporations, and libraries in more than 100 countries use the Internet. Millions of people are users. Yet this extraordinary communication system only came into being in 1969. In the mid-1960s, mainframe computers in research organizations were standalone devices. Computers from different manufacturers were unable to communicate with one another. The Advanced Research Projects Agency (ARPA) in the Department of Defense (DoD) was interested in finding a way to connect computers so that the researchers they funded could share their findings, thereby reducing costs and eliminating duplication of effort. In 1967, at an Association for Computing Machinery (ACM) meeting, ARPA presented its ideas for ARPANET, a small network of connected computers. The idea was that each host computer (not necessarily from the same manufacturer) would be attached to a specialized computer, called an inteiface message processor (IMP). The IMPs, in tum, would be connected to one another. Each IMP had to be able to communicate with other IMPs as well as with its own attached host. By 1969, ARPANET was a reality. Four nodes, at the University of California at Los Angeles (UCLA), the University of California at Santa Barbara (UCSB), Stanford Research Institute (SRI), and the University of Utah, were connected via the IMPs to form a network. Software called the Network Control Protocol (NCP) provided communication between the hosts. In 1972, Vint Cerf and Bob Kahn, both of whom were part of the core ARPANET group, collaborated on what they called the Internetting Projec1. Cerf and Kahn's landmark 1973 paper outlined the protocols to achieve end-to-end delivery of packets. This paper on Transmission Control Protocol (TCP) included concepts such as encapsulation, the datagram, and the functions of a gateway.

Shortly thereafter, authorities made a decision to split TCP into two protocols: Transmission Control Protocol (TCP) and Internetworking Protocol (lP). IP would handle datagram routing while TCP would be responsible for higher-level functions such as segmentation, reassembly, and error detection. The internetworking protocol became known as TCPIIP.

\subsection{Internet Hari Ini}
The Internet has come a long way since the 1960s. The Internet today is not a simple hierarchical structure. It is made up of many wide- and local-area networks joined by connecting devices and switching stations. It is difficult to give an accurate representation of the Internet because it is continually changing-new networks are being added, existing networks are adding addresses, and networks of defunct companies are being removed. Today most end users who want Internet connection use the services of Internet service providers (lSPs). There are international service providers, national service providers, regional service providers, and local service providers. The Internet today is run by private companies, not the government. Figure 1.13 shows a conceptual (not geographic) view of the Internet.

\subsection*{International Internet Service Providers}
At the top of the hierarchy are the international service providers that connect nations together

\subsection*{National Internet Service Providers}
The national Internet service providers are backbone networks created and maintained by specialized companies. There are many national ISPs operating in North America; some of the most well known are SprintLink, PSINet, UUNet Technology, AGIS, and internet Mel. To provide connectivity between the end users, these backbone networks are connected by complex switching stations (normally run by a third party) called network access points (NAPs). Some national ISP networks are also connected to one another by private switching stations called peering points. These normally operate at a high data rate (up to 600 Mbps).

\subsection*{Regional Internet Service Providers}
Regional internet service providers or regional ISPs are smaller ISPs that are connected to one or more national ISPs. They are at the third level of the hierarchy with a smaller data rate.

\subsection*{Local Internet Service Providers}
Local Internet service providers provide direct service to the end users. The local ISPs can be connected to regional ISPs or directly to national ISPs. Most end users are connected to the local ISPs. Note that in this sense, a local ISP can be a company that just provides Internet services, a corporation with a network that supplies services to its own employees, or a nonprofit organization, such as a college or a university, that runs its own network. Each of these local ISPs can be connected to a regional or national service provider.

\section{Protokol dan Standard}
In this section, we define two widely used terms: protocols and standards. First, we define protocol, which is synonymous with rule. Then we discuss standards, which are agreed-upon rules.

\subsection{Prokotol}
In computer networks, communication occurs between entities in different systems. An entity is anything capable of sending or receiving information. However, two entities cannot simply send bit streams to each other and expect to be understood. For communication to occur, the entities must agree on a protocol. A protocol is a set of rules that govern data communications. A protocol defines what is communicated, how it is communicated, and when it is communicated. The key elements of a protocol are syntax, semantics, and timing
\begin{itemize}
  \item[$\odot$] Syntax. The term syntax refers to the structure or format of the data, meaning the order in which they are presented. For example, a simple protocol might expect the first 8 bits of data to be the address of the sender, the second 8 bits to be the address of the receiver, and the rest of the stream to be the message itself. 
  \item[$\odot$] Semantics. The word semantics refers to the meaning of each section of bits. How is a particular pattern to be interpreted, and what action is to be taken based on that interpretation? For example, does an address identify the route to be taken or the final destination of the message? 
  \item[$\odot$] Timing. The term timing refers to two characteristics: when data should be sent and how fast they can be sent. For example, if a sender produces data at 100 Mbps but the receiver can process data at only 1 Mbps, the transmission will overload the receiver and some data will be lost.
\end{itemize}

\subsection*{Standard}
Standards are essential in creating and maintaining an open and competitive market for equipment manufacturers and in guaranteeing national and international interoperability of data and telecommunications technology and processes. Standards provide guidelines to manufacturers, vendors, government agencies, and other service providers to ensure the kind of interconnectivity necessary in today's marketplace and in international communications. Data communication standards fall into two categories: de facto (meaning "by fact" or "by convention") and de jure (meaning "by law" or "by regulation")
\begin{itemize}
  \item[$\odot$] De facto. Standards that have not been approved by an organized body but have been adopted as standards through widespread use are de facto standards. De facto standards are often established originally by manufacturers who seek to define the functionality of a new product or technology. 
  \item[$\odot$] De jure. Those standards that have been legislated by an officially recognized body are de jure standards.
\end{itemize}

\subsection{Standards Organizations}
Standards are developed through the cooperation of standards creation committees, forums, and government regulatory agencies.

\subsection*{Standards Creation Committees}
While many organizations are dedicated to the establishment of standards, data telecommunications in North America rely primarily on those published by the following:

\begin{itemize}
  \item[$\odot$] International Organization for Standardization (ISO). The ISO is a multinational body whose membership is drawn mainly from the standards creation committees of various governments throughout the world. The ISO is active in developing cooperation in the realms of scientific, technological, and economic activity. 
  \item[$\odot$] International Telecommunication Union-Telecommunication Standards Sector (ITU-T). By the early 1970s, a number of countries were defining national standards for telecommunications, but there was still little international compatibility. The United Nations responded by forming, as part of its International Telecommunication Union (ITU), a committee, the Consultative Committee for International Telegraphy and Telephony (CCITT). This committee was devoted to the research and establishment of standards for telecommunications in general and for phone and data systems in particular. On March 1, 1993, the name of this committee was changed to the International Telecommunication UnionTelecommunication Standards Sector (ITU-T). 
  \item[$\odot$] American National Standards Institute (ANSI). Despite its name, the American National Standards Institute is a completely private, nonprofit corporation not affiliated with the U.S. federal government. However, all ANSI activities are undertaken with the welfare of the United States and its citizens occupying primary importance. 
  \item[$\odot$] Institute of Electrical and Electronics Engineers (IEEE). The Institute of Electrical and Electronics Engineers is the largest professional engineering society in the world. International in scope, it aims to advance theory, creativity, and product quality in the fields of electrical engineering, electronics, and radio as well as in all related branches of engineering. As one of its goals, the IEEE oversees the development and adoption of international standards for computing and communications. 
  \item[$\odot$] Electronic Industries Association (EIA). Aligned with ANSI, the Electronic Industries Association is a nonprofit organization devoted to the promotion of electronics manufacturing concerns. Its activities include public awareness education and lobbying efforts in addition to standards development. In the field of information technology, the EIA has made significant contributions by defining physical connection interfaces and electronic signaling specifications for data communication.
\end{itemize}

\subsection*{Forums}
Telecommunications technology development is moving faster than the ability of standards committees to ratify standards. Standards committees are procedural bodies and by nature slow-moving. To accommodate the need for working models and agreements and to facilitate the standardization process, many special-interest groups have developed forums made up of representatives from interested corporations. The forums work with universities and users to test, evaluate, and standardize new technologies. By concentrating their efforts on a particular technology, the forums are able to speed acceptance and use of those technologies in the telecommunications community. The forums present their conclusions to the standards bodies.

\subsection*{Regulatory Agencies}
All communications technology is subject to regulation by government agencies such as the Federal Communications Commission (FCC) in the United States. The purpose of these agencies is to protect the public interest by regulating radio, television, and wire/cable communications. The FCC has authority over interstate and international commerce as it relates to communications

\subsection{Internet Standard}
An Internet standard is a thoroughly tested specification that is useful to and adhered to by those who work with the Internet. It is a formalized regulation that must be followed. There is a strict procedure by which a specification attains Internet standard status. A specification begins as an Internet draft. An Internet draft is a working document (a work in progress) with no official status and a 6-month lifetime. Upon recommendation from the Internet authorities, a draft may be published as a Request for Comment (RFC). Each RFC is edited, assigned a number, and made available to all interested parties. RFCs go through maturity levels and are categorized according to their requirement level

\section{Ringkasan}
\begin{itemize}
  \item[$\odot$] Data communications are the transfer of data from one device to another via some form of transmission medium.
  \item[$\odot$] A data communications system must transmit data to the correct destination in an accurate and timely manner. 
  \item[$\odot$] The five components that make up a data communications system are the message, sender, receiver, medium, and protocol. 
  \item[$\odot$] Text, numbers, images, audio, and video are different forms of information. 
  \item[$\odot$] Data flow between two devices can occur in one of three ways: simplex, half-duplex, or full-duplex. 
  \item[$\odot$] A network is a set of communication devices connected by media links. 
  \item[$\odot$] In a point-to-point connection, two and only two devices are connected by a dedicated link. In a multipoint connection, three or more devices share a link. 
  \item[$\odot$] Topology refers to the physical or logical arrangement of a network. Devices may be arranged in a mesh, star, bus, or ring topology. 
  \item[$\odot$] A network can be categorized as a local area network or a wide area network. 
  \item[$\odot$] A LAN is a data communication system within a building, plant, or campus, or between nearby buildings. 
  \item[$\odot$] A WAN is a data communication system spanning states, countries, or the whole world. 
  \item[$\odot$] An internet is a network of networks. 
  \item[$\odot$] The Internet is a collection of many separate networks. 
  \item[$\odot$] There are local, regional, national, and international Internet service providers. 
  \item[$\odot$] A protocol is a set of rules that govern data communication; the key elements of a protocol are syntax, semantics, and timing.
  \item[$\odot$] Standards are necessary to ensure that products from different manufacturers can work together as expected. 
  \item[$\odot$] The ISO, ITD-T, ANSI, IEEE, and EIA are some of the organizations involved in standards creation. 
  \item[$\odot$] Forums are special-interest groups that quickly evaluate and standardize new technologies. 
  \item[$\odot$] A Request for Comment is an idea or concept that is a precursor to an Internet standard
\end{itemize}

\section{Latihan}

\subsection*{Pertanyaan ulasan}

\begin{enumerate}
  \item Identify the five components of a data communications system. 
  \item What are the advantages of distributed processing? 
  \item What are the three criteria necessary for an effective and efficient network? 
  \item What are the advantages of a multipoint connection over a point-to-point connection? 
  \item What are the two types of line configuration? 
  \item Categorize the four basic topologies in terms of line configuration. 
  \item What is the difference between half-duplex and full-duplex transmission modes? 
  \item Name the four basic network topologies, and cite an advantage of each type. 
  \item For n devices in a network, what is the number of cable links required for a mesh, ring, bus, and star topology? 
  \item What are some of the factors that determine whether a communication system is a LAN or WAN?
  \item What is an internet? What is the Internet? 
  \item Why are protocols needed? 
  \item Why are standards needed?
\end{enumerate}

\subsection*{Latihan}
\begin{enumerate}[resume]
  \item What is the maximum number of characters or symbols that can be represented by Unicode? 
  \item A color image uses 16 bits to represent a pixel. What is the maximum number of different colors that can be represented? 
  \item Assume six devices are arranged in a mesh topology. How many cables are needed? How many ports are needed for each device?
\end{enumerate}