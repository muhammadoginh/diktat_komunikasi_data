\chapter{Data dan Sinyal}
One of the major functions of the physical layer is to move data in the form of electromagnetic signals across a transmission medium. Whether you are collecting numerical statistics from another computer, sending animated pictures from a design workstation, or causing a bell to ring at a distant control center, you are working with the transmission of data across network connections. Generally, the data usable to a person or application are not in a form that can be transmitted over a network. For example, a photograph must first be changed to a form that transmission media can accept. Transmission media work by conducting energy along a physical path

\section{Analog dan Digital}
To be transmitted, data must be transformed to electromagnetic signals. Both data and the signals that represent them can be either analog or digital in form.

\subsection{Analog and Digital Data}
Data can be analog or digital. The term analog data refers to information that is continuous; digital data refers to information that has discrete states. For example, an analog clock that has hour, minute, and second hands gives information in a continuous form; the movements of the hands are continuous. On the other hand, a digital clock that reports the hours and the minutes will change suddenly from 8:05 to 8:06. Analog data, such as the sounds made by a human voice, take on continuous values. When someone speaks, an analog wave is created in the air. This can be captured by a microphone and converted to an analog signal or sampled and converted to a digital signal. Digital data take on discrete values. For example, data are stored in computer memory in the form of Os and 1s. They can be converted to a digital signal or modulated into an analog signal for transmission across a medium.

\subsection{Analog and Digital Signals}
Like the data they represent, signals can be either analog or digital. An analog signal has infinitely many levels of intensity over a period of time. As the wave moves from value A to value B, it passes through and includes an infinite number of values along its path. A digital signal, on the other hand, can have only a limited number of defined values. Although each value can be any number, it is often as simple as 1 and O. The simplest way to show signals is by plotting them on a pair of perpendicular axes. The vertical axis represents the value or strength of a signal. The horizontal axis represents time. Figure 3.1 illustrates an analog signal and a digital signal. The curve representing the analog signal passes through an infinite number of points. The vertical lines of the digital signal, however, demonstrate the sudden jump that the signal makes from value to value

\subsection{Periodic and Nonperiodic Signals}

\section{Sinyal Analog Periodik}
Periodic analog signals can be classified as simple or composite. A simple periodic analog signal, a sine wave, cannot be decomposed into simpler signals. A composite periodic analog signal is composed of multiple sine waves

\subsection{Sine Wave}
The sine wave is the most fundamental form of a periodic analog signal. When we visualize it as a simple oscillating curve, its change over the course of a cycle is smooth and consistent, a continuous, rolling flow. Figure 3.2 shows a sine wave. Each cycle consists of a single arc above the time axis followed by a single arc below it.

\section{Sinyal Digital}

\section{Gangguan Transmisi}
Signals travel through transmission media, which are not petfect. The impetfection causes signal impairment. This means that the signal at the beginning of the medium is not the same as the signal at the end of the medium. What is sent is not what is received. Three causes of impairment are attenuation, distortion, and noise (see Figure 3.25)

\section{Batas Kecepatan Data}

\section{Performa}

\section{Ringkasan}
\begin{itemize}
  \item[$\odot$] Digital-to-analog conversion is the process of changing one of the characteristics of an analog signal based on the information in the digital data
  \item[$\odot$] Digital-to-analog conversion can be accomplished in several ways: amplitude shift keying (ASK), frequency shift keying (FSK), and phase shift keying (PSK). Quadrature amplitude modulation (QAM) combines ASK and PSK.
  \item[$\odot$] In amplitude shift keying, the amplitude of the carrier signal is varied to create signal elements. Both frequency and phase remain constant while the amplitude changes.
  \item[$\odot$] In frequency shift keying, the frequency of the carrier signal is varied to represent data. The frequency of the modulated signal is constant for the duration of one signal element, but changes for the next signal element if the data element changes. Both peak amplitude and phase remain constant for all signal elements.
  \item[$\odot$] In phase shift keying, the phase of the carrier is varied to represent two or more different signal elements. Both peak amplitude and frequency remain constant as the phase changes.
  \item[$\odot$] A constellation diagram shows us the amplitude and phase of a signal element, particularly when we are using two carriers (one in-phase and one quadrature).
  \item[$\odot$] Quadrature amplitude modulation (QAM) is a combination of ASK and PSK. QAM uses two carriers, one in-phase and the other quadrature, with different amplitude levels for each carrier.
  \item[$\odot$] Analog-to-analog conversion is the representation of analog information by an analog signal. Conversion is needed if the medium is bandpass in nature or if only a bandpass bandwidth is available to us
  \item[$\odot$] Analog-to-analog conversion can be accomplished in three ways: amplitude modulation (AM), frequency modulation (FM), and phase modulation (PM).
  \item[$\odot$] In AM transmission, the carrier signal is modulated so that its amplitude varies with the changing amplitudes of the modulating signal. The frequency and phase of the carrier remain the same; only the amplitude changes to follow variations in the information.
  \item[$\odot$] In PM transmission, the frequency of the carrier signal is modulated to follow the changing voltage level (amplitude) of the modulating signal. The peak amplitude and phase of the carrier signal remain constant, but as the amplitude of the information signal changes, the frequency of the carrier changes correspondingly.
  \item[$\odot$] In PM transmission, the phase of the carrier signal is modulated to follow the changing voltage level (amplitude) of the modulating signal. The peak amplitude and frequency of the carrier signal remain constant, but as the amplitude of the information signal changes, the phase of the carrier changes correspondingly.
\end{itemize}

\section{Latihan}

\subsection*{Pertanyaan ulasan}

\begin{enumerate}
  \item Describe the goals of multiplexing.
  \item List three main multiplexing techniques mentioned in this chapter. 
  \item Distinguish between a link and a channel in multiplexing. 
  \item Which of the three multiplexing techniques is (are) used to combine analog signals? Which ofthe three multiplexing techniques is (are) used to combine digital signals? 
  \item Define the analog hierarchy used by telephone companies and list different levels ofthe hierarchy. 
  \item Define the digital hierarchy used by telephone companies and list different levels of the hierarchy. 
  \item Which of the three multiplexing techniques is common for fiber optic links? Explain the reason. 
  \item Distinguish between multilevel TDM, multiple slot TDM, and pulse-stuffed TDM. 
  \item Distinguish between synchronous and statistical TDM. 
  \item Define spread spectrum and its goal. List the two spread spectrum techniques discussed in this chapter. 
  \item Define FHSS and explain how it achieves bandwidth spreading. 
  \item Define DSSS and explain how it achieves bandwidth spreading.
\end{enumerate}

\subsection*{Latihan}
\begin{enumerate}[resume]
  \item Assume that a voice channel occupies a bandwidth of 4 kHz. We need to multiplex 10 voice channels with guard bands of 500 Hz using FDM. Calculate the required bandwidth.
  \item We need to transmit 100 digitized voice channels using a pass-band channel of 20 KHz. What should be the ratio of bits/Hz if we use no guard band?
  \item In the analog hierarchy of Figure 6.9, find the overhead (extra bandwidth for guard band or control) in each hierarchy level (group, supergroup, master group, and jumbo group)
  \item We need to use synchronous TDM and combine 20 digital sources, each of 100 Kbps. Each output slot carries 1 bit from each digital source, but one extra bit is added to each frame for synchronization. Answer the following questions
  \item Repeat Exercise 16 if each output slot carries 2 bits from each source
  \item We have 14 sources, each creating 500 8-bit characters per second. Since only some of these sources are active at any moment, we use statistical TDM to combine these sources using character interleaving. Each frame carries 6 slots at a time, but we need to add four-bit addresses to each slot. Answer the following questions:
  \item Ten sources, six with a bit rate of 200 kbps and four with a bit rate of 400 kbps are to be combined using multilevel TDM with no synchronizing bits. Answer the following questions about the final stage of the multiplexing
  \item Four channels, two with a bit rate of 200 kbps and two with a bit rate of 150 kbps, are to be multiplexed using multiple slot TDM with no synchronization bits. Answer the following questions:
  \item Two channels, one with a bit rate of 190 kbps and another with a bit rate of 180 kbps, are to be multiplexed using pulse stuffing TDM with no synchronization bits. Answer the following questions:
  \item Answer the following questions about a T-1 line:
  \item Show the contents of the five output frames for a synchronous TDM multiplexer that combines four sources sending the following characters. Note that the characters are sent in the same order that they are typed. The third source is silent
  \item Figure 6.34 shows a multiplexer in a synchronous TDM system. Each output slot is only 10 bits long (3 bits taken from each input plus 1 framing bit). What is the output stream? The bits arrive at the multiplexer as shown by the arrows
  \item Figure 6.35 shows a demultiplexer in a synchronous TDM. If the input slot is 16 bits long (no framing bits), what is the bit stream in each output? The bits arrive at the demultiplexer as shown by the arrows
  \item Answer the following questions about the digital hierarchy in Figure 6.23
  \item What is the minimum number of bits in a PN sequence if we use FHSS with a channel bandwidth of B =4 KHz and Bss =100 KHz?
  \item An FHSS system uses a 4-bit PN sequence. If the bit rate of the PN is 64 bits per second, answer the following questions:
  \item A pseudorandom number generator uses the following formula to create a random series:
  \item We have a digital medium with a data rate of 10 Mbps. How many 64-kbps voice channels can be carried by this medium if we use DSSS with the Barker sequence?
\end{enumerate}
